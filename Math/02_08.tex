\documentclass{article}
\RequirePackage{amsmath}
\title{Solving Linear Systems}
\author{Guilherme Novaes Lima}
\date{--  February  --\\-  2023  -}
\begin{document}
\maketitle
\pagebreak
\begin{center}
\textbf{\Large Contents}
\end{center}
\renewcommand{\contentsname}{}
\vspace*{\fill}
\tableofcontents
\vspace*{\fill}
\pagebreak
\section{Initial Analysis}
\subsection{Elements}
When first analyzing a system of linear equations, you can characterize it based in a number of factors listed below.
\[n \rightarrow \text{Number of equations}\]
\[m \rightarrow \text{Number of unknown elements}\]
\subsection{Classifying the system}
After knowing the elements of the system, one can classify it accordingly:
\begin{itemize}
    \item Independent system: Has exactly one solution ($m = n$).
    \item Dependent system: Has infinitely many solutions ($m > n$).
    \item Inconsistent: Has no solution ($m < n$) 
\end{itemize}
\section{Choosing the solution method}
\subsection{Types of methods}
The methods used in solving linear systems can be divided into 2 major groups.
\begin{itemize}
    \item Direct: 
    \item Indirect:
\end{itemize}
\section{Finally... Using the method of choice}
\subsection{Gauss Elimination}
To use this method, one needs to find an equivalent system to the one being solved 
\end{document}